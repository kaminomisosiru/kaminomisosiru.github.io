%% start of file `template.tex'.
%% Copyright 2006-2015 Xavier Danaux (xdanaux@gmail.com), 2020-2022 moderncv maintainers (github.com/moderncv).
%
% This work may be distributed and/or modified under the
% conditions of the LaTeX Project Public License version 1.3c,
% available at http://www.latex-project.org/lppl/.


\documentclass[10pt,a4paper,sans]{moderncv}        % possible options include font size ('10pt', '11pt' and '12pt'), paper size ('a4paper', 'letterpaper', 'a5paper', 'legalpaper', 'executivepaper' and 'landscape') and font family ('sans' and 'roman')

% moderncv themes
\moderncvstyle{casual}                             % style options are 'casual' (default), 'classic', 'banking', 'oldstyle' and 'fancy'
\moderncvcolor{blue}                               % color options 'black', 'blue' (default), 'burgundy', 'green', 'grey', 'orange', 'purple' and 'red'
%\renewcommand{\familydefault}{\sfdefault}         % to set the default font; use '\sfdefault' for the default sans serif font, '\rmdefault' for the default roman one, or any tex font name
\nopagenumbers{}                                  % uncomment to suppress automatic page numbering for CVs longer than one page

% adjust the page margins
\usepackage[scale=0.75]{geometry}
\setlength{\footskip}{90.60005pt}                 % depending on the amount of information in the footer, you need to change this value. comment this line out and set it to the size given in the warning
\setlength{\hintscolumnwidth}{2.7cm}                % if you want to change the width of the column with the dates
%\setlength{\makecvheadnamewidth}{10cm}            % for the 'classic' style, if you want to force the width allocated to your name and avoid line breaks. be careful though, the length is normally calculated to avoid any overlap with your personal info; use this at your own typographical risks...

% font loading
% for luatex and xetex, do not use inputenc and fontenc
% see https://tex.stackexchange.com/a/496643
\ifxetexorluatex
  \usepackage{fontspec}
  \usepackage{unicode-math}
  \defaultfontfeatures{Ligatures=TeX}
  \setmainfont{Latin Modern Roman}
  \setsansfont{Latin Modern Sans}
  \setmonofont{Latin Modern Mono}
  \setmathfont{Latin Modern Math} 
\else
  \usepackage[utf8]{inputenc}
  \usepackage[T1]{fontenc}
  \usepackage{lmodern}
\fi

% document language
\usepackage[english]{babel}  % FIXME: using spanish breaks moderncv

% personal data
\name{Keitaro}{Hashimoto}
% \title{Résumé title}                               % optional, remove / comment the line if not wanted
\born{14 April 1995}                                 % optional, remove / comment the line if not wanted
% \address{street and number}{postcode city}{country}% optional, remove / comment the line if not wanted; the "postcode city" and "country" arguments can be omitted or provided empty
\phone[mobile]{+81~(90)~6014~2574}                   % optional, remove / comment the line if not wanted; the optional "type" of the phone can be "mobile" (default), "fixed" or "fax"
% \phone[fixed]{+2~(345)~678~901}
% \phone[fax]{+3~(456)~789~012}
\email{keitaro.hashimoto000@gmail.com}                               % optional, remove / comment the line if not wanted
\homepage{https://kaminomisosiru.github.io/}                         % optional, remove / comment the line if not wanted

% Social icons
\social[linkedin]{keitaro-hashimoto}                        % optional, remove / comment the line if not wanted
% \social[xing]{john\_doe}                           % optional, remove / comment the line if not wanted

\social[github]{kaminomisosiru}                              % optional, remove / comment the line if not wanted
% \social[gitlab]{jdoe}                              % optional, remove / comment the line if not wanted
% \social[codeberg]{jdoe}                            % optional, remove / comment the line if not wanted
% \social[bitbucket]{jdoe}                           % optional, remove / comment the line if not wanted
% \social[stackoverflow]{0000000/johndoe}            % optional, remove / comment the line if not wanted

% \social[skype]{jdoe}                               % optional, remove / comment the line if not wanted
\social[orcid]{0000-0002-2232-9443}                  % optional, remove / comment the line if not wanted
% \social[researchgate]{jdoe}                        % optional, remove / comment the line if not wanted
% \social[researcherid]{AAQ-6801-2020}                        % optional, remove / comment the line if not wanted
% \social[googlescholar]{smdIoD4AAAAJ}            % optional, remove / comment the line if not wanted

% \social[twitter]{ji\_doe}                          % optional, remove / comment the line if not wanted
% \social[mastodon]{mastodon.social/web/@user}       % optional, remove / comment the line if not wanted
% \social[telegram]{jdoe}                            % optional, remove / comment the line if not wanted
% \social[whatsapp]{12345678901}                     % optional, remove / comment the line if not wanted
% \social[signal]{12345678901}                       % optional, remove / comment the line if not wanted
% \social[matrix]{@johndoe:matrix.org}               % optional, remove / comment the line if not wanted
% \social[discord]{jdoe\#0000}                       % optional, remove / comment the line if not wanted




\extrainfo{Nationality: Japan}                 % optional, remove / comment the line if not wanted
\photo[64pt][0.4pt]{picture.jpg}                       % optional, remove / comment the line if not wanted; '64pt' is the height the picture must be resized to, 0.4pt is the thickness of the frame around it (put it to 0pt for no frame) and 'picture' is the name of the picture file
% \quote{Some quote}                                 % optional, remove / comment the line if not wanted

% bibliography adjustments (only useful if you make citations in your resume, or print a list of publications using BibTeX)
%   to show numerical labels in the bibliography (default is to show no labels)
%\makeatletter\renewcommand*{\bibliographyitemlabel}{\@biblabel{\arabic{enumiv}}}\makeatother
% \renewcommand*{\bibliographyitemlabel}{[\arabic{enumiv}]}
%   to redefine the bibliography heading string ("Publications")
%\renewcommand{\refname}{Articles}

% bibliography with mutiple entries
\usepackage{multibib}
\newcites{journal,conference,misc}{{Journals},{Conferences},{Others}}
%----------------------------------------------------------------------------------
%            content
%----------------------------------------------------------------------------------
\begin{document}
%\begin{CJK*}{UTF8}{gbsn}                          % to typeset your resume in Chinese using CJK
%-----       resume       ---------------------------------------------------------
\makecvtitle

\section{Education}
\cventry{2020--2023}{Ph.D.}{Tokyo Institute of Technology}{Tokyo, Japan}{}{Post-quantum key exchange protocols for secure messaging. Supervised by Wakaha Ogata (Tokyo Institute of Technology)}  % arguments 3 to 6 can be left empty
\cventry{2018--2020}{Master of Engineering}{Tokyo Institute of Technology}{Tokyo, Japan}{}{Major: Information and communication engineering, specialized in cryptography}  % arguments 3 to 6 can be left empty
\cventry{2014--2018}{Bachelor of Engineering}{Tokyo Institute of Technology}{Tokyo, Japan}{}{Major: Computer sciences}

% \section{Master thesis}
% \cvitem{title}{\emph{Title}}
% \cvitem{supervisors}{Supervisors}
% \cvitem{description}{Short thesis abstract}

\section{Experience}
\cventry{04/2023--Now}{Researcher}{National Institute of Advanced Industrial Science and Technology (AIST)}{Tokyo, Japan}{}{}
\cventry{04/2022--03/2023}{JSPS Research Fellowship for Young Scientists}{Japan Society for the Promotion of Science}{Tokyo, Japan}{}{}
\cventry{07/2022}{Visiting internship}{PQShield SAS}{Paris, France}{}{}
\cventry{06/2020--03/2023}{Research Assistant}{National Institute of Advanced Industrial Science and Technology (AIST)}{Tokyo, Japan}{}{}
\cventry{08/2018--09/2018}{Summer internship}{Nippon Telegraph and Telephone Corporation (NTT)}{Tokyo, Japan}{}{}
\cventry{08/2017--09/2017}{Summer internship}{Infosec Corporation}{Tokyo, Japan}{}{}

\section{Teaching}
\cventry{08/2019}{Teaching Assistant}{Tokyo Institute of Technology}{Tokyo, Japan}{}{Teaching Assistant in the exchange Summer School with Zhejiang University}
\cventry{04/2019--08/2019}{Teaching Assistant}{Tokyo Institute of Technology}{Tokyo, Japan}{}{Teaching Assistant in the C Programming class and the Experiments on embedded systems class}
\cventry{06/2018--08/2018}{Teaching Assistant}{Tokyo Institute of Technology}{Tokyo, Japan}{}{Teaching Assistant in the C Programming class}


% Publications from a BibTeX file without multibib
%  for numerical labels: \renewcommand{\bibliographyitemlabel}{\@biblabel{\arabic{enumiv}}}% CONSIDER MERGING WITH PREAMBLE PART
%  to redefine the heading string ("Publications"): \renewcommand{\refname}{Articles}
% \nocite{*}
% \bibliographystyle{plain}
% \bibliography{./bib/conference}                        % 'publications' is the name of a BibTeX file

% Publications from a BibTeX file using the multibib package
\section{Publications}
\nocitejournal{*}
\bibliographystylejournal{alpha}
\bibliographyjournal{./bib/journal}                   % 'publications' is the name of a BibTeX file
\nociteconference{*}
\bibliographystyleconference{alpha}
\bibliographyconference{./bib/conference}                   % 'publications' is the name of a BibTeX file
\nocitemisc{*}
\bibliographystylemisc{alpha}
\bibliographymisc{./bib/misc}                   % 'publications' is the name of a BibTeX file

\section{Talks}
\subsection{International conference talks}
\cventry{11/2022}{ACM CCS}{How to Hide MetaData in MLS-Like Secure Group Messaging: Simple, Modular, and Post-Quantum}{Los Angeles, USA}{}{}
\cventry{11/2021}{ACM CCS}{A Concrete Treatment of Efficient Continuous Group Key Agreement via Multi-Recipient PKEs}{Virtual}{}{}
\cventry{05/2021}{PKC}{An Efficient and Generic Construction for Signal's Handshake (X3DH): Post-Quantum, State Leakage Secure, and Deniable}{Virtual}{}{}

\subsection{Invited talks}
\cventry{09/2022}{Workshop on Cryptography and Information Security (WCIS)}{A Concrete Treatment of Efficient Continuous Group Key Agreement via Multi-Recipient PKEs}{Virtual}{}{}
\cventry{07/2022}{Talk at ENS de Lyon}{A Concrete Treatment of Efficient Continuous Group Key Agreement via Multi-Recipient PKEs}{Lyon, France}{}{}
\cventry{09/2021}{SCIS/CSS Invited Session in IWSEC}{Design and Implementation of a Post-Quantum Authenticated Key Exchange Protocol for Signal}{Virtual}{}{}

\section{Languages}
% \cvcomputer{Japanese}{Native}{English}{Intermediate}
\cvitemwithcomment{Japanese}{Native}{}
\cvitemwithcomment{English}{Intermediate}{}

\section{Certifications}
\cventry{03/2021}{Improve Your English Communication Skills Specialization}{Coursera}{A3ZGXJ8RWW5T}{}{}
\cventry{03/2021}{Introdu ction to Mathematical Thinking}{Coursera}{WQY3UEVLZSEE}{}{}
\cventry{12/2015}{Applied Information Technology Engineer}{Ministry of Economy, Trade and Industry}{AP-2015-10-03112}{}{}
\cventry{10/2014}{Fundamental Information Technology Engineer}{Ministry of Economy, Trade and Industry}{FE-2014-10-04834}{}{}


\section{Computer skills}
\cvcomputer{Programming}{Java, Rust, Python}{Typesetting}{\LaTeX/\TeX}
% \cvitem{Programming}{Java, Rust, Python}
% \cvitem{Typesetting}{\LaTeX/\TeX}

% \section{Skill matrix}
% \cvitem{Skill matrix}{Alternatively, provide a skill matrix to show off your skills}
% %% Skill matrix as an alternative to rate one's skills, computer or other. 

% %% Adjusts width of skill matrix columns. 
% %% Usage \setcvskillcolumns[<width>][<factor>][<exp_width>]
% %% <width>, <exp_width> should be lengths smaller than \textwidth, <factor> needs to be between 0 and 1.
% %% Examples:
% % \setcvskillcolumns[5em][][]%    adjust first column. Same as \setcvskillcolumns[5em]
% % \setcvskillcolumns[][0.45][]%   adjust third (skill) column. Same as \setcvskillcolumns[][0.45]
% % \setcvskillcolumns[][][\widthof{``Year''}]%     adjust fourth (years) column.
% % \setcvskillcolumns[][0.45][\widthof{``Year''}]%
% % \setcvskillcolumns[\widthof{``Languag''}][0.48][]
% % \setcvskillcolumns[\widthof{``Languag''}]%

% %% Adjusts width of legend columns. Usage \setcvskilllegendcolumns[<width>][<factor>]
% %% <factor> needs to be between 0 and 1. <width> should be a length smaller than \textwidth
% %% Examples:
% % \setcvskilllegendcolumns[][0.45]
% % \setcvskilllegendcolumns[\widthof{``Legend''}][0.45]
% % \setcvskilllegendcolumns[0ex][0.46]% this is usefull for the banking style

% %% Add a legend if you are using \cvskill{<1-5>} command or \cvskillentry
% %% Usage \cvskilllegend[*][<post_padding>][<first_level>][<second_level>][<third_level>][<fourth_level>][<fifth_level>]{<name>}
% % \cvskilllegend % insert default legend without lines
% \cvskilllegend*[1em]{}% adjust post spacing
% % \cvskilllegend*{Legend}%  Alternatively add a description string
% %% adjust the legend entries for other languages, here German
% % \cvskilllegend[0.2em][Grundkenntnisse][Grundkenntnisse und eigene Erfahrung in Projekten][Umfangreiche Erfahrung in Projekten][Vertiefte Expertenkenntnisse][Experte\,/\,Spezialist]{Legende}

% %% Alternative legend style with the first three skill levels in one column
% %% Usage \cvskillplainlegend[*][<post_padding>][<first_level>][<second_level>][<third_level>][<fourth_level>][<fifth_level>]{<name>}
% % \setcvskilllegendcolumns[][0.6]%  works for classic, casual, banking
% % \setcvskilllegendcolumns[][0.55]%  works better for oldstyle and fancy
% % \cvskillplainlegend{}
% % \cvskillplainlegend[0.2em][Grundkenntnisse][Grundkenntnisse und eigene Erfahrung in Projekten][Umfangreiche Erfahrung in Projekten][Vertiefte Expertenkenntnisse][Experte/Guru]{Legende}

% %% Add a head of the skill matrix table with descriptions.
% %% Usage \cvskillhead[<post_padding>][<Level>][<Skill>][<Years>][<Comment>]%
% \cvskillhead[-0.1em]%   this inserts the standard legend in english and adjust padding
% %% Adjust head of the skill matrix for other languages
% % \cvskillhead[0.25em][Level][F\"ahigkeit][Jahre][Bemerkung]

% %% \cvskillentry[*][<post_padding>]{<skill_cathegory>}{<0-5>}{<skill_name>}{<years_of_experience>}{<comment>}% 
% %% Example usages:
% \cvskillentry*{Language:}{3}{Python}{2}{I'm so experienced in Python and have realised a million projects. At least.}
% \cvskillentry{}{2}{Lilypond}{14}{So much sheet music! Man, I'm the best!}
% \cvskillentry{}{3}{\LaTeX}{14}{Clearly I rock at \LaTeX}
% \cvskillentry*{OS:}{3}{Linux}{2}{I only use Archlinux btw}% notice the use of the starred command and the optional 
% \cvskillentry*[1em]{Methods}{4}{SCRUM}{8}{SCRUM master for 5 years}
% %% \cvskill{<0-5>} command
% % \cvitem{\textbackslash{cvskill}:}{Skills can be visually expressed by the \textbackslash{cvskill} command, e.g. \cvskill{2}}

% \section{Interests}
% \cvitem{hobby 1}{Description}
% \cvitem{hobby 2}{Description}
% \cvitem{hobby 3}{Description}

% \section{Extra 1}
% \cvlistitem{Item 1}
% \cvlistitem{Item 2}
% \cvlistitem{Item 3. This item is particularly long and therefore normally spans over several lines. Did you notice the indentation when the line wraps?}

% \section{Extra 2}
% \cvlistdoubleitem{Item 1}{Item 4}
% \cvlistdoubleitem{Item 2}{Item 5\cite{book2}}
% \cvlistdoubleitem{Item 3}{Item 6. Like item 3 in the single column list before, this item is particularly long to wrap over several lines.}

\section{References}
% \begin{cvcolumns}
%   \cvcolumn{Name}{\begin{itemize}\item Person 1\item Person 2\item Person 3\end{itemize}}
%   \cvcolumn{Category 2}{Amongst others:\begin{itemize}\item Person 1, and\item Person 2\end{itemize}(more upon request)}
%   \cvcolumn[0.5]{All the rest \& some more}{\textit{That} person, and \textbf{those} also (all available upon request).}
% \end{cvcolumns}
\cvlistitem{Wakaha Ogata (Ph.D. adviser): ogata.w.aa@m.titech.ac.jp}
% \begin{cvcolumns}
%   \cvcolumn{Name}{}
%   % \cvcolumn{Function}{\begin{itemize}\item Ph.D. adviser\end{itemize}}
%   % \cvcolumn{E-mail address}{\begin{itemize}\item ogata.w.aa@m.titech.ac.jp\end{itemize}}
% \end{cvcolumns}

\end{document}


%% end of file `template.tex'.

